\begin{comment}
direct editing on remote HPC machines. Students can choose editors such {\em vi}, {\em vim}, {\em nano}, {\em pico}, or {\em emacs}. The latter, {\em emacs}, has been used predominantly by us as we can teach it in five minutes while focusing on the ten most important commands. If more sophisticated commands are needed, the emacs reference card is a valuable helper~\cite{emacs-reference}. As the same shortcuts are used in a terminal, we reduced the complexity of teaching students using a command line editor as well as editing in the terminal with convenient shortcuts. These CLI tools are vital to apprentices of machine learning as they save time in the time-sensitive programs in which they are conducting research work.
\emnd{comment}



The use of simple editors such as {\em Notepad++}, {\em IDLE}, or {\em nano} is insufficient as they do not support the best software engineering practices that can be achieved with an advanced programming framework development environment. Instead, students benefit from advanced tools such as {\em PyCharm} or {\em Visual Studio Code} (vscode) as they provide sophisticated features to improve code quality and also provide strong integration with git. One of the strengths of PyCharm is that it has a 

